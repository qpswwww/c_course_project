% !Mode:: "TeX:UTF-8"

%%% 此部分需要自行填写: (1) 中文摘要及关键词 (2) 英文摘要及关键词
%%%%%%%%%%%%%%%%%%%%%%%%%%%%%

%%% 郑重声明部分无需改动

%%%---- 郑重声明 (无需改动)------------------------------------%
%\newpage
%\vspace*{20pt}
%\begin{center}{\ziju{0.8}\textbf{\songti\zihao{2} 郑重声明}}\end{center}
%\par\vspace*{30pt}
%\renewcommand{\baselinestretch}{2}

%{\zihao{4}%

%本人呈交的学位论文, 是在导师的指导下, 独立进行研究工作所取得的成果,
%所有数据、图片资料真实可靠. 尽我所知, 除文中已经注明引用的内容外,
%本学位论文的研究成果不包含他人享有著作权的内容.
%对本论文所涉及的研究工作做出贡献的其他个人和集体,
%均已在文中以明确的方式标明. 本学位论文的知识产权归属于培养单位.\\[2cm]

%\hspace*{1cm}本人签名: $\underline{\hspace{3.5cm}}$
%\hspace{2cm}日期: $\underline{\hspace{3.5cm}}$\hfill\par}
%------------------------------------------------------------------------------
\baselineskip=23pt  % 正文行距为 23 磅
%------------------------------------------------------------------------------





%%======中文摘要===========================%
\begin{cnabstract}
  手写数字分类问题是机器学习与模式识别领域的基础问题之一,
  也是OCR(光学字符识别)的核心技术,具有重要的现实意义。
  本文针对MNIST手写数字数据集,通过C++实现了基于多层感知机网络
  的手写数字数据集分类算法,并采用交叉验证、小批量随机梯度
  下降算法、扩充训练数据集的方法优化算法,
  最终达到了98.042\%的分类准确率。


\end{cnabstract}
\par
\vspace*{2em}


%%%%--  关键词 -----------------------------------------%%%%%%%%
%%%%-- 注意: 每个关键词之间用“;”分开,最后一个关键词不打标点符号
\cnkeywords{多层感知机; 手写数字; 分类; 优化; }


%%====英文摘要==========================%


\begin{enabstract}
  The classification of handwritten digits is one of the basic problems of machine learning and pattern recognition, 
  which is also the key technology of optical character recognition(OCR). 
  In this paper, according to the MNIST handwritten digits dataset, 
  we implement a classifying algorithm based on multi-layer perceptron(MLP) by C++, 
  and optimize it by cross-validation, mini-batch stochastic gradient descent algorithm 
  and expanding the training dataset. 
  The accuracy of our algorithm reaches 98.042\%.

\end{enabstract}
\par
\vspace*{2em}

%%%%%-- Key words --------------------------------------%%%%%%%
%%%%-- 注意: 每个关键词之间用“;”分开,最后一个关键词不打标点符号
 \enkeywords{Multi-layer Perceptron; Handwritten Digits; Classification; Optimization;}
